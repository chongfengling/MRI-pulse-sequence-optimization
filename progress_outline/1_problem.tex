\section{Introduction and Problem Statement}
\subsection{What is the problem?}
Pulse sequences design in Magnetic Resonance Imaging (MRI) refers to the process of creating and optimizing a series of RadioFrequency (RF) pulses, magnetic field gradients and a static magnetic field to achieve a desired imaging goal. The interaction of these sequences manipulates the behavior of the nuclear spins in the body, which in turn produces a signal that can be used to form the basis of an image.

\subsection{Why is this problem important?}
The design of pulse sequences is a critical aspect of MRI, as it directly determines the quality of the image and acquisition time. Furthermore, the design of pulse sequences is various according to specific imaging applications and is constrained by the hardware and searching space of parameters \citep{0438}. Achieving a consistent and accurate image requires more than simply relying on default parameters. Instead, it necessitates the development of a model that takes into account the limitations of both the theoretical framework and realistic constraints involved in the imaging process.

\subsection{What is the goal of this project?}
The primary aim of this project is to create a Reinforcement Learning based model to optimize gradient echo sequence subject to some constraints. Specifically, the task involves designing a new pulse sequence generator that is capable of producing an optimized signal compared to a typical gradient-echo sequence-based signal, while conforming to certain constraints such as the slew rate of gradient coils.