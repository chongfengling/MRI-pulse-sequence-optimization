\section{Plan and Challenges}
The first phase of the plan can be divided into two primary parts: theoretical knowledge acquisition and programming practice. Theoretical knowledge acquisition encompasses a comprehensive review of the MRI principles, ranging from pulse sequence design to image reconstruction, as well as an in-depth study of reinforcement learning theories. The programming practice involves replicating a classical reinforcement learning case study and reproducing the results presented in the aforementioned literature \citep{0438}. This phase is estimated to about one month.
\\\\
Upon mastery of the fundamental theoretical knowledge and code architecture, the second phase will predominantly focus on incorporating constraints related to the slew rate of gradient coils. The third phase entails further expansion of the constraints. The duration of the latter two phases is contingent upon the outcomes of the first phase, and thus, the estimation of the time required is currently unattainable.

\subsection{Timeline (From Week 14 to week 34)}
\subsubsection{Phase 1 (6 weeks)} 
\begin{itemize}
    \item w14-w16: Review of MRI principles and pulse sequence design
    \item w14-w16: Study reinforcement learning theories
    \item w16-w17: Implementation of a classical reinforcement learning case
    \item w17-w19: Reproduction of the results presented in the literature \citep{0438}
\end{itemize}
\textbf{Outcome}
\begin{itemize}
    \item w14: A note for MRI (I).
    \item w16: A note for MRI (II).
    \item w17: Code for the classical reinforcement learning case.
    \item w19: Code for literature reproduction.
\end{itemize}

\subsubsection{Phase 2 (6 weeks)}
\begin{itemize}
    \item w??: Incorporation of constraints related to the slew rate of gradient coils
\end{itemize}

\subsubsection{Phase 3 (6 weeks)}
\begin{itemize}
    \item w??: Further expansion of the constraints
\end{itemize}